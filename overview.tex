
\begin{overview}[-1pt] % remove the first % and adjust the number in square braces to shift the overview on the page
Much of the rich world has only slowly recovered from the output slump that started in 2009. But some of our problems go back even further. Productivity growth started to slow in the early 2000s, and has been slow all around the world. 

Australia has done remarkably well, with per capita output up about 30 per cent on 2009, while the US is at +15 and Europe at +5. Luck, policy, China. 

But we do have a fairly crap non-traded economy. Low productivity growth, lots of fat happy firms, costs consumers X. And ?? slows innovation? And slow non-mining investment

What then should govt do to avoid stagnation, and what to do to reignite dynamism?

Avoiding stagnation seems mostly a matter of macro policy. And it is true that supportive monetary policy and a credible path to balance are key. 

But reigniting dynamism is mostly about 3 major steps. 

\begin{itemize}
    \item Investment: ensure demand is strong. Avoid excesses (prudential). 
    \item Concentration: make prices easier to compare, access, charge banks for equity. (something about choke point assets?). Possible support with lobbying changes? 
    \item Innovation: employee share options? IP - ensure do not subscribe to toughest ( ?)
\end{itemize}

Taking these steps could increase non-traded productivity by \$ X billion within a decade, and offer the following other benefits.  

\end{overview}